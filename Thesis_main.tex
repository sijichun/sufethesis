%% LyX 2.1.4 created this file.  For more info, see http://www.lyx.org/.
%% Do not edit unless you really know what you are doing.
\documentclass[unicode,pdfcover]{sufethesis}
\usepackage{fontspec}
\usepackage{color}
\usepackage{array}
\usepackage{longtable}
\usepackage{graphicx}
\usepackage[unicode=true,
 bookmarks=true,bookmarksnumbered=true,bookmarksopen=false,
 breaklinks=false,pdfborder={0 0 1},backref=false,colorlinks=true]
 {hyperref}
\hypersetup{pdftitle={博士论文标题},
 pdfauthor={你的名字},
 pdfsubject={华南理工大学博士学位论文},
 pdfkeywords={关键字1, 关键字2},
 unicode=false,linkcolor=blue, anchorcolor=black, citecolor=olive, filecolor=magenta, menucolor=red, urlcolor=magenta, pdfstartview=FitH}

\makeatletter

%%%%%%%%%%%%%%%%%%%%%%%%%%%%%% LyX specific LaTeX commands.
\providecommand{\LyX}{\texorpdfstring%
  {L\kern-.1667em\lower.25em\hbox{Y}\kern-.125emX\@}
  {LyX}}
%% Because html converters don't know tabularnewline
\providecommand{\tabularnewline}{\\}

\makeatother

\usepackage{xunicode}
\begin{document}

\title{上海财经大学博士毕业论文LaTex\& Lyx模板}


\author{司继春}


\supervisor{指导教师:周亚虹\ 副教授}


\institute{上海财经大学}


\date{2016年8月1日}

\maketitle
\frontmatter
\begin{abstractCN}
本文主要介绍了上海财经大学博士学位论文\LaTeX 和\LyX{}模板的使用方法。本模板根据华南理工大学博士学位论文\href{https://github.com/alwintsui/scutthesis}{模板}修改制作而成,在此表示感谢。
\end{abstractCN}

\keywordsCN{\textbf{毕业论文;上海财经大学;\textbf{\LaTeX};\LyX{}}}
\begin{abstractEN}
The english abstract.
\end{abstractEN}

\keywordsEN{\textbf{Thesis; SUFE; \textbf{\LaTeX}; \LyX{}}}

\tableofcontents{}

\mainmatter


\chapter{引言}


\section{选题背景与意义}

为方便上海财经大学博士毕业生使用\LaTeX 和\LyX{}完成毕业论文,将自己写毕业论文时的模板开源。本模板基于华南理工大学博士学位论文\href{https://github.com/alwintsui/scutthesis}{模板}修改制作而成。如果您对这个模板有改进,请fork
and pull。感谢。


\section{研究的主要创新之处}

市面上第一个上海财经大学的博士毕业论文模板。


\chapter{文献综述}


\section{代码来源}

基础模板来源于华南理工大学博士学位论文\href{https://github.com/alwintsui/scutthesis}{模板},并根据上海财经大学学位论文的要求进行了相应的修改。


\chapter{使用方法}


\section{基本使用方法}

对于\LaTeX{}用户,请使用根目录下的Thesis\_main.tex相应修改即可,并使用xelatex命令进行编译。对于\LyX{}用户,修改Thesis\_main.lyx即可。


\section{格式修改需求}

格式定义在sufethesis.cls文件中,可以根据自己的需求修改此文件。在不确定的情况下不建议直接修改此文件。


\section{字体}

本模板在mac下编译通过,如果使用其他系统(如Linux、Windows等),需要修改sufethesis.cls中的字体为相应的系统字体:
\begin{quotation}
\textbackslash{}setCJKmainfont\{Songti SC\}

\textbackslash{}setCJKfamilyfont\{song\}\{Songti SC\}

\textbackslash{}setCJKfamilyfont\{hei\}\{Heiti SC\} 

\textbackslash{}setCJKfamilyfont\{kai\}\{Kaiti SC\} 

\textbackslash{}setCJKfamilyfont\{fang\}\{STFangsong\}

\textbackslash{}newcommand\{\textbackslash{}songti\}\{\textbackslash{}CJKfamily\{song\}\} 

\textbackslash{}newcommand\{\textbackslash{}heiti\}\{\textbackslash{}CJKfamily\{hei\}\} 

\textbackslash{}newcommand\{\textbackslash{}kaiti\}\{\textbackslash{}CJKfamily\{kai\}\} 

\textbackslash{}newcommand\{\textbackslash{}fangsong\}\{\textbackslash{}CJKfamily\{fang\}\}
\end{quotation}
将「Songti SC」等分别修改为系统对应的字体即可。\textbf{不修改字体会导致编译不通过}。


\section{公式录入}

公式输入与\LaTeX{}及\LyX{}使用方法一致,如$y=x'\beta+\epsilon$,以及:
\[
\hat{\beta}=\left(X'X\right)^{-1}X'Y
\]
此外公式可以进行编号:
\begin{equation}
\hat{\beta}_{iv}=\left[X'Z\left(Z'Z\right)^{-1}Z'X\right]^{-1}\left[X'Z\left(Z'Z\right)^{-1}Z'Y\right]
\end{equation}
公式编号自动按照章进行分组。


\section{定理声明}

可以声明定理等,如:
\begin{theorem}
OLS估计为最优无偏线性估计量(BLUE),且$\hat{\beta}\overset{p}{\rightarrow}\beta$。\end{theorem}
\begin{proof}
略。
\end{proof}


\section{图片、表格插入}

图片和表格编号实例:

\begin{table}


\caption{表格示例}


\begin{centering}
\begin{tabular}{ccc}
\hline 
 & (1) & (2)\tabularnewline
\hline 
labor & $\underset{(0.2)}{0.7}${*}{*} & $\underset{(0.2)}{0.7}${*}{*}\tabularnewline
capital & $\underset{(0.1)}{0.3}${*}{*} & $\underset{(0.1)}{0.3}${*}{*}\tabularnewline
Constant & $\underset{(0.5)}{1.5}${*}{*} & $\underset{(0.5)}{1.5}${*}{*}\tabularnewline
\hline 
\multicolumn{3}{l}{{\small{}注:仅为示意性展示}}\tabularnewline
\hline 
\end{tabular}
\par\end{centering}

\end{table}


\begin{figure}
\begin{centering}
\includegraphics[scale=0.4]{figures/1}
\par\end{centering}

\caption{图示例}


\end{figure}



\section{参考文献}

参考文献可以使用Bib\TeX{}进行管理,覆盖sufethesis.bib即可。


\chapter{结论}

Enjoy it.

\bibliographystyle{sufethesis}
\nocite{*}
\bibliography{sufethesis}



\chapterx{致谢}

感谢之前的代码提供者,特别是华南理工大学博士学位论文\href{https://github.com/alwintsui/scutthesis}{模板}的制作者。

\begin{minipage}[t]{0.8\columnwidth}%
\begin{flushright}
司继春
\par\end{flushright}

\begin{flushright}
2016年8月1日
\par\end{flushright}%
\end{minipage}


\chapterx{攻读博士学位期间取得的研究成果}

已发表(包括已接受待发表)的论文,以及已投稿、或已成文打算投稿、或拟成文投稿的论文情况(只填写与学位论文内容相关的部分):

\begin{table}
\begin{longtable}{|>{\centering}m{0.5cm}|>{\centering}m{2.3cm}|>{\centering}m{3.5cm}|>{\centering}m{2.6cm}|>{\centering}m{2cm}|>{\centering}m{1.3cm}|>{\centering}m{0.9cm}|}
\hline 
序号 & 作者(全体作者,按顺序排列) & 题 目 & 发表或投稿刊物名称、级别 & 发表的卷期、年月、页码 & 相当于学位论文的哪一部分(章、节) & 被索引收录情况\tabularnewline
\hline 
1 &  &  &  &  &  & \tabularnewline
\hline 
 &  &  &  &  &  & \tabularnewline
\hline 
 &  &  &  &  &  & \tabularnewline
\hline 
 &  &  &  &  &  & \tabularnewline
\hline 
 &  &  &  &  &  & \tabularnewline
\hline 
 &  &  &  &  &  & \tabularnewline
\hline 
\end{longtable}
\end{table}

\end{document}
